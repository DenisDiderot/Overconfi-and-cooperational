\documentclass[11pt,twoside,a4paper,leqno]{article}
\usepackage{a4wide}
\usepackage[T1]{fontenc}
\usepackage[utf8]{inputenc}
\usepackage{float, afterpage, rotating, graphicx}
\usepackage{longtable, booktabs, tabularx}
\usepackage{verbatim}
\usepackage{eurosym, calc, chngcntr}
\usepackage{amsmath, amssymb, amsfonts, amsthm, bm, delarray} 
\usepackage{setspace}
\usepackage[unicode=true]{hyperref}
\usepackage[margin=0.70in]{geometry}
%\usepackage[backend=biber, natbib=true, bibencoding=inputenc, bibstyle=authoryear-ibid, citestyle=authoryear-comp, maxnames=10]{biblatex}
% \bibliography{bib/hmg}

\hypersetup{colorlinks=true, linkcolor=black, anchorcolor=black, citecolor=black, filecolor=black, menucolor=black, runcolor=black, urlcolor=black}
\setlength{\parskip}{.5ex}
\setlength{\parindent}{0ex}
\onehalfspacing
\graphicspath{{ .\ }}

\begin{document}
The design of my experiment relies on the tradition of the public good games, with a modification that allows to have a clearly different behavior for under- and overconfident individuals. In Stage 1 the players are faced with a certain amount of questions, IQ test like, and they have to answer to them first and subsequently estimate how many of them were correct. I am going to use the “lottery method” suggested in Blavatskyy (2008)\textsuperscript{\cite{Blav}} and used by Goette et al. (2015)\textsuperscript{\cite{Goette}}, as it directly assesses the overconfidence problem within an environment where incentive compatibility is granted. Players will hence form beliefs about their own absolute ability and can then estimate their relative ability among the pool of subjects in their session, namely the percentile they occupy, which will be crucial in the second stage. This could even be explicitly asked, in order to have a better comprehension in the analysis regarding the rest of the experiment. In my opinion it does not need to be incentivised, as it is likely going to rely on the estimation of Stage 1 and additionally it is in their self interest for the upcoming Task to estimate as good as possible their position. \newline
In Stage 2 the players will be randomly mixed in couples, but this does not so far seem to present any problem in being open to groups as well. Players in the couples will play a public good game, where the marginal per capita return (hereafter MPCR) depends on their ability multiplied by some factor called $\alpha$. Before all of this, the participants will try the mechanism a couple of times, with different MPCRs (also bigger than 1), in order to be sure that effects due to previous participation to similar games or to studies that entails some notions of game theory are smoothened as much as possible. 
Players will be told in the instructions at the beginning that the payoff will be structured as follows:
\begin{align}
	U_{i} = E_{i} - c_{i} + \alpha(\sum_{i=1}^N q_{i}^A)(\sum_{i=1}^N c_{i})
\end{align}
where $E_{i}$ is the initial endowment, $c_{i}$ the contribution of player i and $q_{i}^A$ the actual percentile in which the two players ranked in the first stage. Participants will be told that $\alpha$ might change during the course of the experiment and it will be communicated to them afterwards. This will vary and it could make the total MPCR appear as even bigger than one, hence making it rationally efficient to invest in the public good game. The intention was to create a mechanism, where the contribution in a common project could appear individually rational. Therefore we need to define the beliefs about the utility of the players:
\begin{align}
	E[U_{i}] = E_{i} - c_{i} + \alpha(\sum_{i=1}^N q_{i}^B)(\sum_{i=1}^N c_{i})
\end{align}
where $q_{i}^B$ are the beliefs about the ranking of all the participants at his session. Notice that the believed MPCR is consequently defined as $E[MPCR] = \alpha(\sum_{i=1}^N q_{i}^B)$. Not knowing where the $\alpha$ comes from, they will just think the payoff being related to their (and others') ability multiplied by some unknown factor. If they have correct beliefs, they will exactly perceive the MPCR as equal to 1, meaning that even the selfish individual is indifferent between investing in the public good or keeping the money by himself, as both will have the same return. Moreover it would be socially optimal to invest. If the sum is smaller than alpha, perceived MPCR is going to be smaller than 1, with the prediction for a selfish individual to just keep everything by himself, and if bigger, $MPCR > 1$, hence they will think being in this case individually rational to invest in the public good. The advantage of shaping the MPCR with the $\alpha$ is that it also allows to have a clear prediction for those players who are poorly skilled but overconfident, whereas, with $\alpha$ simply equal to one, the contribution is optimal only if the confidence is sufficiently high. Clearly this does not give us the chance to make any prediction related to the own confidence at the moment, because we are considering the sum of the $q_{i}$, but some modifications to the game can be done. A treatment that has a direct prediction for the overconfident is the one that tells the players all $q_{j}^A$, where $j \neq i$; knowing the others' actual ability, the perceived MPCR of the participant will depend only on the belief about his own ability. Clearly if $q^A < q^B$, then he sees the incentive - regardless of his attitude towards his social group - to invest all the money in the public good, which is also socially efficient. With this mechanism, the overconfidence bias can be "positively" exploited to enhance cooperation. \newline
Some problems might arise with the MPCR being one and fluctuating in expectations around it. An idea could be to multiply the $\alpha$ by 0.4, as in Fischbacher et al. paper\textsuperscript{\cite{Fisch}}, with the MPCR being $0.4*(\sum_{i=1}^N q_{i}^A)/\alpha$. For acceptable levels of overconfidence the perceived MPCR should not become larger than 1. Here clearly for the overconfident it is not going to appear strictly convenient to invest in the public good, but he will still think he is facing a higher MPCR, therefore having an higher incentive to invest.\newline
Another interesting treatment is the one where the participant is informed about his own percentile, but not about the others'. This can show us if the overconfidence about himself can also be transferred towards the group, as Li et al.(2016)\textsuperscript{\cite{Li}} suggest. Another treatment could include the possibility to communicate to the others their own ability, opening to the possibility of strategic deception. Here an open question is whether I need or not to guarantee incentive compatibility by making this declaration somehow costly. Surely by declaring a higher level of confidence, if successful, the individual might convince the other to invest more, as it appears more likely to have an $MPCR > 1$, despite the real monetary incentive stays untouched. This treatment could then investigate the possible role of social preferences in the overconfidence: individuals see a possible incentive in declaring a higher ability than the actual one. \newline
It is interesting to notice here how this public good belongs to the class of the homogeneous returns public goods, but can also present some elements from the literature of the uncertain heterogeneous public good games. Participants do realize that the MPCR is going to be the same for everyone, but they also realize that everyone will have different beliefs about it and will act accordingly. In the heterogeneous returns literature (Fisher et. al 1995\textsuperscript{\cite{Fisher}}, Tan 2008\textsuperscript{\cite{Tan}}) people tend to have higher contribution levels if they have higher returns, which an overconfident individual might think he is facing. This is straightforward, as the higher the productivity of a subject, the less effort will be required to contribute to the public good and, even more extremely, it is going to be strictly optimal if the returns are bigger than one-fold the investment. I could use an instrument introduced by Cartwright and Lovett (2014)\textsuperscript{\cite{Cart}}, the so-called \textit{cooperation factor}, calculated as follows: $\theta _{i} = covar(L,c_{i}(L))/var(L)$. L in their case was the "leader" contribution, but it can safely be the average contribution by the members in the group and $c_{i}$ is the contribution by the i individual. As Falk and Fischbacher (2006)\textsuperscript{\cite{Falk}} stated the cooperation factor of the conditional cooperators is expected to increase with the MPCR, whereas their proportion not. This could actually be contradicted by my design, where it would be optimal convenient to invest in the public good if overconfident. Similarly to what Tan did in his experiment I could then look at the difference between those who are in the symmetric treatment (which in my case means the people who have correct beliefs and $MPCR=1$) and those who are in an asymmetric one, i.e. those who are under- or overconfident.


\clearpage

\begin{thebibliography}{9}

\bibitem{Blav}
  Blavatskyy, Pavlo R.
  (2009),
  \emph{Betting on own knowledge: experimental test of overconfidence},
  Journal of Risk and Uncertainty.
  \textbf{38} (1):
  pp. 39-49.

\bibitem{Cart}
  Cartwright E.J. and Lovett D.
  (2014),
  \emph{Conditional Cooperation and the Marginal per Capita Return in
Public Good Games},
  Games.
  \textbf{5}:
  pp. 234-256.

\bibitem{Falk}
  Falk A. and Fischbacher U.
  (2006),
  \emph{A theory of reciprocity},
  Games and Economic Behavior.
  \textbf{54}:
  pp. 293-315.

\bibitem{Fisch}
  Fischbacher U., Gätcher S. and Fehr E.
  (2001),
  \emph{Are people conditionally cooperative? Evidence from a public goods experiment},
  Economics Letters.
  \textbf{71} (3):
  pp. 397-404.
  
\bibitem{Fisher}
  Fisher, Joseph et al.
  (1995),
  \emph{Heterogenous demand for public goods: Behavior in the voluntary contributions mechanism},
  Public Choice.
  \textbf{85}:
  pp. 249-266.

\bibitem{Goette}
  Goette L. et al.
  (2015),
  \emph{Stress pulls us apart. Anxiety leads to differences in competitive confidence under stress},
  Psychoneuroendocrinology.
  \textbf{54}:
  pp. 115-123.
  
\bibitem{Li}
  Li, Jianbiao and Yin, Xile and Bao, Te
  (2016),
  \emph{Does Overconfidence Promote Cooperation? Experimental Evidence from a Threshold Public Goods Game},
  Available at SSRN: \url{https://ssrn.com/abstract=2766826}
  
\bibitem{Tan}
  Tan, Fangfang
  (2008),
  \emph{Punishment in a linear public good game with productivity heterogeneity},
  De Economist.
  \textbf{156}:
  pp. 269-293.


  
\end{thebibliography}
\end{document}
