\documentclass[11pt,twoside,a4paper,leqno]{article}
\usepackage{a4wide}
\usepackage[T1]{fontenc}
\usepackage[utf8]{inputenc}
\usepackage{float, afterpage, rotating, graphicx}
\usepackage{longtable, booktabs, tabularx}
\usepackage{verbatim}
\usepackage{eurosym, calc, chngcntr}
\usepackage{amsmath, amssymb, amsfonts, amsthm, bm, delarray} 
\usepackage{setspace}
\usepackage[unicode=true]{hyperref}
\usepackage[margin=0.60in]{geometry}

%\usepackage[backend=biber, natbib=true, bibencoding=inputenc, bibstyle=authoryear-ibid, citestyle=authoryear-comp, maxnames=10]{biblatex}
% \bibliography{bib/hmg}

\hypersetup{colorlinks=true, linkcolor=black, anchorcolor=black, citecolor=black, filecolor=black, menucolor=black, runcolor=black, urlcolor=black}
\setlength{\parskip}{.5ex}
\setlength{\parindent}{0ex}
\onehalfspacing
\graphicspath{{ .\ }}
\pagenumbering{gobble}

\begin{document}
\title{{\Huge \textbf{Research Proposal}}\\[0.5 cm]Overconfidence and Cooperation:\\an ambiguous relationship}
\date{}
\author{Author: Pietro Trentini \and Supervisor: Dr. Lorenz Goette}
\maketitle

Overconfidence is a persistent phenomenon in our modern society. Its biggest strength stays in the ability of convincing others of being qualitatively superior than the average, however there is more than meets the eye: overconfidence often changes the behavior of the individual himself on many levels. These are tightly linked to its origins, which are still discussed in the literature with three main motivations identified: consumption value, motivation value and signaling value\textsuperscript{\cite{Ben}}\textsuperscript{\cite{Burks}}. I decided to focus my research on the latter aspect, being the one I found more convincing, as it roots the origin of this bias in the social realm. I would like to investigate the effects of overconfidence in a public good game, building on the work of Li et al.\textsuperscript{\cite{Li}}, where individuals’ returns will depend on their own and others’ ability. The idea is to resemble multiple real world situations in which returns depend on unobservable abilities - both of the individual and his group mates - which can be overly rosy and thus influence the contribution choice. Especially in firms and organizations the return of the performance is often tightly linked to unknown ability and the contribution can increase when there is an incentive for the smarter or the overconfident individuals to contribute more. One interesting example could also be that of the many open source communities, which rely on individual's contributions and on their ability. In this sense high ability and overconfidence have the same prediction, with individuals cooperating significantly. Moreover, the literature agrees on the overconfident individuals to take more risk than average, and in this sense team productivity in such setup can be enhanced.  On the other side part of the literature associates overconfidence with a sense of superiority and self-sufficiency, which could harm the provision of the public good. Therefore I think that shedding some light on this controversial relationship is due and I would like to contribute with my master thesis to this academic discussion. 

\section*{Experimental Design}
The design of my experiment relies on the tradition of the public good games, with the marginal personal return (hereafter MPCR) depending on the beliefs about both ability of the individual and his group companions. General intellectual ability is tested in Stage 1 of my experiment, using the Bochum Matrices Test (BOMAT)\textsuperscript{\cite{BOMAT}}. Clearly my intention is not to claim it to be an unambiguous measure of intelligence, but just to use it as an instrument for the elicitation of beliefs about own ability. This part of the experiment will have a countdown of 10 minutes, during which the participants will try to solve correctly as many matrices as possible. When the time expires they are going to be asked to elicit their confidence about the result, relatively to another randomly drawn participant, using the strategy method.\textsuperscript{\cite{Blav}}\textsuperscript{\cite{Goette}} The elicitation of confidence is therefore incentivized for the subject and it results incentive compatible. Afterwards they are introduced to Stage 2, where they face a two-player public good game and the following utility function:
\begin{align}
	U_{i} = E_{i} - c_{i} + \alpha(q_{i}^A + q_{-i}^A)(c_{i} + c_{-i})
\end{align}
where $E_{i}$ is the initial endowment, $c_{i}$ the contribution of player i and $q^A$ the actual percentile in which the two players ranked in the first stage, which is not known. Therefore we need to define the beliefs about the utility of the players:
\begin{align}
	E[U_{i}] = E_{i} - c_{i} + \alpha(q_{i}^B + q_{-i}^B)(c_{i} + c_{-i})
\end{align}
where $q^B$ are the beliefs about the ranking of all the participants at his session. Notice that the believed MPCR is consequently defined as $E[MPCR] = \alpha(q_{i}^B + q_{-i}^B)$. Participants will be told before the beginning of Stage 2 the value of the $\alpha$, which can be either $\alpha = 0.5$ or $\alpha= 0.8$. From multiple sources in the literature \textsuperscript{\cite{KD}}\textsuperscript{\cite{Burks}}\textsuperscript{\cite{Healy}}, on average individuals have beliefs about their ability around $q_{i}^B = 0.65$. This can give us exact predictions about the efficiency and the individual rationality of investing in the public good for an overconfident individual. In the first treatment, disregarding how big the overconfidence of the individual is about his and the other's ability, he will face an expected $E[MPCR] < 0.5*(1.0 + 1.0)$, hence $E[MPCR] < 1$. Thus the public good game will never appear as individually rational, whereas even a below-average overconfident individual, with $q_{i}^B = 0.5$, will perceive the game as team-efficient. The latter treatment however guarantees that for an overconfident individual, with $q_{i}^B > 0.65$ the game will even appear as individually rational, while efficiency requires only small levels of self-confidence. In both treatments, the higher the self-confidence, the higher should be the contribution.
Beside this between-participant treatment there is a within-participant treatment dimension that regards the information that is given to the players, with the following conditions:
\begin{itemize}
	\item The individual knows $q_{-i}^A$, the actual percentile in which the other player in the group is located;
	\item He knows $q_{-i}^B$, namely the belief the other participant has about herself;
	\item He is not informed about any beliefs
	\item A control treatment where $MPCR = \alpha$
\end{itemize}
The players will face 12 rounds, 3 per each information treatment, and they will be asked at the end of each of them to elicit their beliefs about $q_{-i}^A$ and $c_{-i}$. This latter elicitation will not be incentivized. The information treatments will be randomized on the two-person group level and they do not seem to have any issue regarding order or clustering effects. Only one round will be chosen for the payment.

\section*{Participants, Payments and the Lab}
I plan to run four sessions, on two distinct days: the first day a session that will serve as a pilot, with the other three following on another day. The desired number of participants is 16 per each session, in order to have exactly 4 participants in each session per information treatment. Two additional participants will be invited to the laboratory in order to ensure at least an even number of participants, and if possible a multiple of 8. The expected payment can be found in Table 1. The expected timeline of the experiment is the following: 20 minutes for the allocation of the seats and the instruction of Stage 1, 10 for the completion of the IQ test, 10 for the elicitation of preferences, 10 minutes for the instruction of Stage 2, 15 for the completion of Stage 2 and another 15 for the follow-up survey and the payment, for a total duration of 80-90 minutes.


\clearpage 
\begin{thebibliography}{9}
\footnotesize

\bibitem{Ben}
  Benabou R. and Tirole J.
  (2002),
  \emph{Self-Confidence and Personal Motivation},
  The Quarterly Journal of Economics.
  \textbf{117} (3):
  pp. 871-915.

\bibitem{Blav}
  Blavatskyy, Pavlo R.
  (2009),
  \emph{Betting on own knowledge: experimental test of overconfidence},
  Journal of Risk and Uncertainty.
  \textbf{38} (1):
  pp. 39-49.
  
\bibitem{Burks}
  Burks S.V., Carpenter J.P., Goette L., Rustichini A.
  (2013),
  \emph{Overconfidence and Social Signalling},
  Review of Economic Studies.
  \textbf{80}:
  pp. 949-983.

%  
\bibitem{KD}
  Kruger J. and Dunning D.
  (1999),
  \emph{Unskilled and Unaware of It: How Difficulties in Recognizing One's Incompetence Lead to Inflated Self-Assessments},
  Journal of Personality and Social Psychology.
  \textbf{77} (6):
  pp. 1121-1134.

\bibitem{Goette}
  Goette L. et al.
  (2015),
  \emph{Stress pulls us apart. Anxiety leads to differences in competitive confidence under stress},
  Psychoneuroendocrinology.
  \textbf{54}:
  pp. 115-123.

\bibitem{Healy}
  Healy A. and Pate J.G.
  (2007),
  \emph{Overconfidence, Social Groups, and Gender: Evidence from the Lab and Field},
  Available at SSRN: \url{https://ssrn.com/abstract=934320}


\bibitem{BOMAT}
  Hossiep R., Turck. D., Hasella M.
  (1999),
  \emph{Bochumer Matrizentest (BOMAT) Advanced},
  Hogrefe.
  
    
\bibitem{Li}
  Li, Jianbiao and Yin, Xile and Bao, Te
  (2016),
  \emph{Does Overconfidence Promote Cooperation? Experimental Evidence from a Threshold Public Goods Game},
  Available at SSRN: \url{https://ssrn.com/abstract=2766826}
  

%
%
%  
\end{thebibliography}

\begin{table}[h!]  
	\centering 
    \caption{Estimation of costs}
	\vspace{5mm}
    \begin{tabular}{ r  r }
    Participants (per session) & 16+2 \\ 
    Participants(total) & 72 \\ \hline
    Expected costs & \\ 
    Show-up Fees & 72*4 = 288 \euro \\ 
    Payment (Stage 1) & 64*5 = 320 \euro \\ 
    Payment (Stage 2) & 64*6 = 384 \euro \\ \hline
    Total & 992 \euro \\
    \end{tabular}\\
\end{table}


\end{document}
